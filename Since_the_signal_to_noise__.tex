

Since the signal to noise ratio of single receiver function trace is low because of the small P-S refraction coefficient, traces with overlapping ray path are usually stacked together to enhance the signal. Usually common conversion point (CCP) stacking is considered as the standard procedure for receiver function imaging, in which individual receiver function is applied moveout correction to zero offset, and then stacked due to the location of their piercing point at different depth. However, CCP stacking method has difficulties to reconstruct complex geometry of the refractors if lateral heterogeneity is significant or the imaged interface is dipping steeply [Ryberg and Weber, 2000]. In order to improve image quality, We adopted the pre-stack Kirchhoff migration [Wilson and Aster, 2005] to deal with the complex geometry of interfaces. In this method, the imaged space is treated as a grid of point scatters, and each receiver function is rescaled and back-projected to all the possible scattering points. A 1D reference model with 75km crustal thickness is used for ray tracing and converting the delay time into depth. After the summation of all receiver functions, we normalized the amplitude of output grid for better illumination. The normalization is only applied in depth domain and each horizontal location in the profile is normalized independently. 
Besides 14 boardband stations that we operated in this region for almost one year and half, we also collected data from 1 INDEPTH II station, 9 INDEPTH III stations and 6 years data from permanent station LSA (shown as Figure 1). Receiver functions were then estimated using teleseismic events with epicenter distance from 30°to 90°, and magnitude over 5.5. After carefully visual selection, totally 1335 receiver function waveforms with high signal to noise ratio were selected from PKU stations, 27 from Indepth II stations, 88 from Indepth III stations, and 51 from Lhasa station.
