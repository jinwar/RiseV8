
Yadong-Gulu rift is one of the longest extensional rifts in the plateau. It extends from the high Himalaya, across Yalu-Zangbu suture, and to east of Namucuo lake around 31$^\circ$N. West of the rift's central-northern section locates The Nyainqentanglha massif as the footwall of the extensional normal fault Figure~\ref{fig:topo}. 
The central part of the rift (Yangbajain portion) initiated normal faulting around 8 Ma ago while northern part (Gulu portion) started at around 5 Ma \cite{Harrison_1995,stockli2002late}. The thermochronological results suggest Nyainqentanglha massif has been uplifted approximately 12-17km since then \cite{Harrison_1995,Kapp_2005}. Total horizontal extension accumulated on this rift system depends on the dipping angle of the normal faults, which is still under debate. With a 15km uplift, horizontal extension may range from 9km on a high-angle (60°) to 21km on a possible low-angle (35°) detachment fault [Cogan et al., 1998]. 
In order to understand how the influence of the rift system on the lithosphere structure of Tibetan Plateau, from January 2007 to May 2008, totally 14 boardband seismic stations of Peking University (PKU) were operated across central-northern Yadong-Gulu rift and Nyainqentanglha massif. By combining data from previous experiments (Indepth II and III), we provide fine image of crustal structure using P-S convert phase.