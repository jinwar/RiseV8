\section{Introduction}

Tibetan plateau is created by the Indian-Eurasian continental collision, which started around 50 Myr ago \cite{Patriat_1984}. The 2000-km north-south convergence between the two continents uplifts the highest plateau with an average elevation of 4 km. The collision not only thicken the plateau crust to around 70 km \cite[e.g.][]{Nabelek_2009}, but also induces clockwise crustal flow around the Eastern Himalayan Syntaxis \cite{Gan_2007}. The eastwards migration of material may relate to the development of the widespread north-south trending rifts in southern Tibet \cite[e.g.][]{Yin_2000,Zhang_2013}. 
Several seismic surveys (INDEPTH I - IV, HICLIMB, etc.) have been conducted on the Tibetan Plateau to explore the tectonic process of the collision. Workers publish many observations and interpretations based on these surveys, e.g. the underthrusting of the Indian plate crust \cite{Zhao_1993}, the existence of partial melt in the plateau’s mid-lower crust \cite{Brown_1996}, the possible downwelling or subduction of the Indian lithosphere \cite[e.g.][]{Tilmann_2003,Li_2008,Nabelek_2009}, the detailed mantle flow pattern from the seismic anisotropy observations \cite[e.g.][]{Hirn_1995,Huang_2000,Fu_2008}.

Most of these studies are focused on the north-south lithospheric shortening and the plateau uplift mechanism, yet some studies highlight the north-south trending, regularly spaced rifts, which is probably created by the east-west extension of Tibetan plateau \cite[e.g.][]{Molnar_1978,Armijo_1986,Yin_2000,Kapp_2004,Kapp_2008}. The orientations of these rifts are north-south in general, with a systematical clockwise rotation from northwest to northeast from western Tibet to eastern Tibet \cite{Kapp_2004}, and the spacing between the rifts decrease from 191km south of Indus-Yalu suture to 101 km in central Tibet \cite{Yin_2000}. These rifts started to develop in Southern Tibet from 18 Myr to 13 Myr \cite{Williams_2001}, which is believed to be the time when the stress setting in Tibet Plateau transit from compression to extension \cite{Kapp_2005}.

Although the north-south variation dominates the Tibet Plateau lithospheric structure, the west-east heterogeneity is significant across the rifts \cite{Zhang_2005}. Unfortunately, most of the seismic surveys in the southern and central Tibet follows these north-south trending rifts, mostly due to the transportation limitation. As a result, the seismic data are biased to the rift structures instead of average plateau structure, which may induce some biases for the interpretation and analysis if the lithosphere structure is significantly different.

Yadong-Gulu rift is one of the longest extensional rifts in the plateau. It extends from the high Himalaya, across Yalu-Zangbu suture, and to east of Namucuo lake around 31$^\circ$N. West of the rift's central-northern section locates The Nyainqentanglha massif as the footwall of the extensional normal fault \ref{fig:topo}. 
The central part of the rift (Yangbajain portion) initiated normal faulting around 8 Ma ago while northern part (Gulu portion) started at around 5 Ma \cite{Harrison_1995}\cite{stockli2002late}. Thermochronological results suggest Nyainqentanglha massif has been uplifted approximately 12-17km since then [Harrison et al., 1995; Kapp et al., 2005]. Total horizontal extension accumulated on this rift system depends on the dipping angle of the normal faults, which is still under debate. With a 15km uplift, horizontal extension may range from 9km on a high-angle (60°) to 21km on a possible low-angle (35°) detachment fault [Cogan et al., 1998]. 
In order to understand how the influence of the rift system on the lithosphere structure of Tibetan Plateau, from January 2007 to May 2008, totally 14 boardband seismic stations of Peking University (PKU) were operated across central-northern Yadong-Gulu rift and Nyainqentanglha massif. By combining data from previous experiments (Indepth II and III), we provide fine image of crustal structure using P-S convert phase.

