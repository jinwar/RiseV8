\section{Introduction}

Tibetan plateau is created by the Indian-Eurasian continental collision, which started around 50 Myr ago \cite{Patriat_1984}. The 2000-km north-south convergence between the two continents uplifts the highest plateau with an average elevation of 4 km. The collision not only thicken the plateau crust to around 70 km \cite[e.g.][]{Nabelek_2009}, but also induces clockwise crustal flow around the Eastern Himalayan Syntaxis \cite{Gan_2007}. The eastwards migration of material may relate to the development of the widespread north-south trending rifts in southern Tibet \cite[e.g.][]{Yin_2000,Zhang_2013}. 
Several seismic surveys (INDEPTH I - IV, HICLIMB, etc.) have been conducted on the Tibetan Plateau to explore the tectonic process of the collision. Workers publish many observations and interpretations based on these surveys, e.g. the underthrusting of the Indian plate crust \cite{Zhao_1993}, the existence of partial melt in the plateau’s mid-lower crust \cite{Brown_1996}, the possible downwelling or subduction of the Indian lithosphere \cite[e.g.][]{Tilmann_2003,Li_2008,Nabelek_2009}, the detailed mantle flow pattern from the seismic anisotropy observations \cite[e.g.][]{Hirn_1995,Huang_2000,Fu_2008}.

Most of these studies are focused on the north-south lithospheric shortening and the plateau uplift mechanism, yet some studies highlight the north-south trending, regularly spaced rifts, which is probably created by the east-west extension of Tibetan plateau \cite[e.g.][]{Molnar_1978,Armijo_1986,Yin_2000,Kapp_2004,Kapp_2008}. The orientations of these rifts are north-south in general, with a systematical clockwise rotation from northwest to northeast from western Tibet to eastern Tibet \cite{Kapp_2004}, and the spacing between the rifts decrease from 191km south of Indus-Yalu suture to 101 km in central Tibet \cite{Yin_2000}. These rifts started to develop in Southern Tibet from 18 Myr to 13 Myr \cite{Williams_2001}, which is believed to be the time when the stress setting in Tibet Plateau transit from compression to extension \cite{Kapp_2005}.

Although the north-south variation dominates the Tibet Plateau lithospheric structure, the west-east heterogeneity is significant across the rifts \cite{Zhang_2005}. Unfortunately, most of the seismic surveys in the southern and central Tibet follows these north-south trending rifts, mostly due to the transportation limitation. As a result, the seismic data are biased to the rift structures instead of average plateau structure, which may induce some biases for the interpretation and analysis if the lithosphere structure is significantly different.

